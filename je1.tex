%iffalse
\let\negmedspace\undefined
\let\negthickspace\undefined
\documentclass[journal,12pt,onecolumn]{IEEEtran}
\usepackage{cite}
\usepackage{amsmath,amssymb,amsfonts,amsthm}
\usepackage{algorithmic}
\usepackage{graphicx}
\usepackage{textcomp}
\usepackage{xcolor}
\usepackage{txfonts}
\usepackage{listings}
\usepackage{enumitem}
\usepackage{mathtools}
\usepackage{gensymb}
\usepackage{comment}
\usepackage{multicol}
\usepackage[breaklinks=true]{hyperref}
\usepackage{tkz-euclide} 
\usepackage{listings}
\usepackage{gvv}                                        
%\def\inputGnumericTable{}                                 
\usepackage[latin1]{inputenc}                                
\usepackage{color}                                            
\usepackage{array}                                            
\usepackage{longtable}                                       
\usepackage{calc}                                             
\usepackage{multirow}                                         
\usepackage{hhline}                                           
\usepackage{ifthen}                                           
\usepackage{lscape}
\usepackage{tabularx}
\usepackage{array}
\usepackage{float}


\newtheorem{theorem}{Theorem}[section]
\newtheorem{problem}{Problem}
\newtheorem{proposition}{Proposition}[section]
\newtheorem{lemma}{Lemma}[section]
\newtheorem{corollary}[theorem]{Corollary}
\newtheorem{example}{Example}[section]
\newtheorem{definition}[problem]{Definition}
\newcommand{\BEQA}{\begin{eqnarray}}
\newcommand{\EEQA}{\end{eqnarray}}
\newcommand{\define}{\stackrel{\triangle}{=}}
\theoremstyle{remark}
\newtheorem{rem}{Remark}

% Marks the beginning of the document
\begin{document}
\bibliographystyle{IEEEtran}
\vspace{3cm}

\title{Session-02-01-2023-shift-1-16-30}
\author{AI24BTECH11018 - Sreya}
\maketitle
\bigskip
\renewcommand{\thefigure}{\theenumi}
\renewcommand{\thetable}{\theenumi}
\begin{enumerate}
\item Let the image of the point P\brak{2,-1,3} in the plane $x+2y-z=0$ be $Q$. Then the distance of the plane 
$3x + 2y + z + 29 = 0$ from the point $Q$ is
\begin{enumerate}
    \item $\frac{22\sqrt{2}}{7}$
    \item $\frac{24\sqrt{2}}{7}$
    \item $2\sqrt{14}$
    \item $3\sqrt{14}$
\end{enumerate}
\item Let $f\brak{x}=\begin{pmatrix}
1+\sin^2{x} & \cos^2{x} & \sin 2x \\
\sin^2{x} & 1+\cos^2{x} & \sin 2x \\
\sin^2{x} & \cos^2{x} & 1+\sin 2x
\end{pmatrix}$, $x\in [\frac{\pi}{6},\frac{\pi}{3}]$.If $\alpha$ a $\beta$ respectively are the maximum and the minimum values of f,then
\begin{enumerate}
    \item $\beta^2-2\sqrt{\alpha}=\frac{19}{4}$
    \item $\beta^2+2\sqrt{\alpha}=\frac{19}{4}$
    \item $\alpha^2 + \beta^2 = 4\sqrt{3}$
    \item $\alpha^2 + \beta^2 = \frac{9}{2}$
\end{enumerate}
\item Let $f\brak{x}=2x+\tan^{-1}x$ and $g\brak{x}=\log_{e}\brak{\sqrt{1+x^2}+x}$, $x\in [0,3]$ then 
\begin{enumerate}
    \item There exists $x\in [0,3]$ such that $f^{'}\brak{x} \textless g^{'}\brak{x}$
    \item max $f\brak{x} \textgreater max g\brak{x}$
    \item There exists $0 \textless x_1 \textless x_2\textless 3$ such that $f\brak{x} \textless g\brak{x}$, $\forall x\in \brak{x_1,x_2}$
    \item min $f^{'}\brak{x}$=$1$+max $ g^{'}\brak{x}$
\end{enumerate}
\item The mean and variance of $5$ observations are $5$ and 
$8$ respectively. If $3$ observations are $1, 3, 5$, then 
the sum of cubes of the remaining two 
observations is
\begin{enumerate}
    \item $1072$
    \item $1792$
    \item $1216$
    \item $1456$
\end{enumerate}
\item The area enclosed by the closed curve C given by 
the differential equation 
$\frac{dy}{dx}+\frac{x+a}{y-2}=0$, $y\brak{1}=0$ is $4\pi$
 Let $P$ and $Q$ be the points of intersection of the 
curve $C$ and the y-axis. If normals at P and Q on 
the curve $C$ intersect x-axis at points $R$ and $S$ 
respectively, then the length of the line segment 
$RS$ is
\begin{enumerate}
    \item $2\sqrt{3}$
    \item $\frac{2\sqrt{3}}{3}$
    \item $2$
    \item $\frac{4\sqrt{3}}{3}$
\end{enumerate}
\item Let $a1 = 8, a_2, a_3,\cdots  a_n$ be an A.P. If the sum of its 
first four terms is $50$ and the sum of ts last four 
terms is $170$, then the product of its middle two 
terms is
\item A\brak{2, 6, 2}, B\brak{-4, 0, \lambda}, C\brak{2, 3, -1} and D\brak{4, 5, 0}, 
$\abs \lambda \leq 5$ are the vertices of a quadrilateral $ABCD$. If 
its area is $18$ square units, then $5-6\lambda$ is equal to
\item The number of 
$3-digit$ numbers, that are divisible 
by either $2$ or $3$ but not divisible by $7$ is
\item The remainder when $19^{200}$ + $23^{200}$ is divided by $49$
is 
\item if$\int_{0}^{1}\brak{x^{21}+x^{14}+x^7}\brak{2x^{14}+3x^{7}+6}^\frac{1}{6}$ where $l,m,n \in m$ and $n$ are co primes then l+m+n is equal to
\end{enumerate}
\end{document}i
